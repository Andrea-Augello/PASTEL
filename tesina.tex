% !TeX spellcheck = it_IT
\documentclass[]{article}
\usepackage{color, soul}
\newcommand{\hlc}[2][yellow]{ {\sethlcolor{#1} \hl{#2}} }
\usepackage[T1]{fontenc}
\usepackage[utf8]{inputenc}
\usepackage[margin=1in]{geometry}




% Title Page
\title{PASTEL}
\author{{Augello Andrea} \and {Bafumo Francesco} \and{La Martina Marco}}


\begin{document}
\maketitle
\tableofcontents


\section{Introduzione}
PASTEL è stato pensato come un linguaggio di scripting per gestire e coordinare le interazioni tra dispositivi IoT basati su TCP.\\
L'utilizzo di PASTEL può essere vantaggioso in sistemi ciberfisici complessi che necessitano una conoscenza dello stato globale del sistema per coordinarsi. Infatti il sitema potrebbe includere dispositivi con risorse limitate e scarse capacità computazionali, che quindi non sono in grado di memorizzare lo stato del resto dei sensori ed effettuare decisioni complesse. 

Un altro contesto in cui PASTEL può essere utile è testare in modo replicabile il corretto comportamento interattivo di un sistema ciberfisico: il linguaggio proposto infatti può facilmente simulare l'output di molti sensori ed inviarlo agli attuatori.

Il resto della \hl{tesina} è strutturato come segue: la sezione~\ref{section:stato-arte} illustra il contesto in cui si colloca il linguaggio proposto, nella sezione~\ref{section:descrizione} vengono analizzati i dettagli implementativi, nella sezione~\ref{section:caratteristiche-linguaggio} si illustrano le caratteristiche di PASTEL e degli esempi di applicazioni, infine nella sezione~\ref{section:conclusioni} vengono riepilogati i punti principali del lavoro svolto.

\section{Stato dell'arte}\label{section:stato-arte}

I dispositivi IoT comunemente eseguono protocolli semplici come CoAP, REST e MQTT~\cite{tandale2017empirical}. \hlc[cyan]{[Inserire descrizione dei protocolli, magari qualche altra citazione]}

Il linguaggio proposto può facilmente essere utilizzato con i protocolli precedenti, eccetto CoAP poiché basato su UDP. L'ambiente target principale però è quello dei dispositivi in grado di eseguire codice simbolico~\cite{gaglio2017dc4cd} e su questi si incentreranno di casi d'uso nella sezione \ref{subsection:casi-d_uso}.  

[...]L'equivalente di \texttt{expect}~\cite{libes1991expect, libes1990expect} per reti di sensori wireless (WSN).

\section{Descrizione del progetto}\label{section:descrizione}
\subsection{Analisi dei requisiti}
\subsection{Scelte progettuali}
\section{Caratteristiche del linguaggio}\label{section:caratteristiche-linguaggio}
\subsection{Grammatica}
\subsection{Descrizione del parser}
\subsection{Casi d'uso}\label{subsection:casi-d_uso}
\subsection{Risultati ottenuti}\label{subsection:risultati}
\section{Conclusioni}\label{section:conclusioni}




\bibliographystyle{unsrt}
\bibliography{references}

\end{document}          
